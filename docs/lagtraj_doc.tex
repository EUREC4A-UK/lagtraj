\documentclass[a4paper,11pt]{article}

\usepackage{amsmath}
\usepackage{amssymb}
\usepackage{natbib}
\usepackage{graphicx}
\usepackage[a4paper,left=2cm,right=2cm,top=2cm,bottom=2cm]{geometry}

\title{Lagtraj: a tool for calculation LES forcings along trajectories}

\begin{document}

\maketitle

DRAFT VERSION

\section*{Abstract}

In recent years, Large-Eddy Simulations that span a length of
multiple days have become increasingly popular. These simulations are
often performed in the context of observational campaigns, or to study
the development of convection over multiple days at observational
supersites. Their main purpose is to simulate the behaviour of
convection and associatied process (microphysics, radiation) at the
scale of metres to a few hundred kilometres.

Lagtraj is a novel framework for calculation of the large-scale
forcings that these simulations require based on forecast or reanalysis
data, in some cases complemented by observational constraints. It aims
to streamline the workflow that is required to perform routine
Large-Eddy Simulation and sensitivity studies based on realistic cases,
and ensure simulations are set up in a traceable fashion. Lagtraj has
been developed to serve as a community tool that can be further
extended with different input and output formats.

\section{Introduction}

The development of Large-Eddy Simulation (LES) for atmospheric science
dates back to \cite{deardorff1970}. LES differs from coarser-scale
convection-permitting models in explicitly resolving the largest
turbulent structures in the atmospheric boundary layer. Due to the
computational cost, LES was initially used for short case studies on
small domains. Initial conditions and forcings were often initialised
in order to make it easier to set up a simulation.

With the growth of computing power over recent decades, it has become
possible to perform high-resolution convection-permitting
modelling and LES corresponding to longer time windows in observational
campaigns, such as TWP-ICE \citep{fridlind2012}. Such studies are
particularly useful where detailed observations about clouds or
turbulence have been made. Another novel application of LES is within a
continuous framework around an observational super-site. Model
evaluation at super-sites has proven a useful strategy for systematic
exploration of parametrisations in NWP models \cite{neggers2012}.
Routine LES operation in this context is useful both for testing the
LES itself (for example, its surface parametrisation and its
representation of clouds), and to help improve and evaluate
parameterisations in NWP and climate models \cite{schalkwijk2015},
\cite{laar2019},\cite{gustafson2020}.

For modelling of case studies over land, often a Eulerian frame of
reference is preferred, both because the observations are made around a
single location and because the height of the ground level can vary
along a flow trajectory. However, for convection over the ocean, it can
be useful to follow air in boundary layer or low clouds in a Lagrangian
framework. This is particularly relevant for capturing the development
of cloud organisation and transitions between cloud regimes. Examples
include \cite{bretherton1999}, \cite{roode2016}, \cite{tomassini2017},
\cite{mohrmann2019}, \cite{neggers2019} and recent work from the NOAA group
(add).

The details of the exact setup matter, see e.g. \cite{smalley2019}.
Replication of observationally-based LES setups, which is particularly
important when comparing diffent models, can be difficult. Results tend
to be sensitive to exact details of setup, such as subsidence
\citep{hohenegger2013,kurowski2020}, advective heating and moistening
tendencies, and the implementation of surface fluxes
\citep{stevens2001}.

The time investment to reproduce the forcings can be an obstacle to
participation in a model intercomparison. For replication, it is
important that the workflow to produce model setups is traceable. The
use of a fully automated framework and the retention of metadata
regarding its configuration are key here. The use of the NetCDF format
(rather than a text-based format) makes it easy to store this metadata.
The forcings for LES intercomparison studies are often derived through
an iterative process. In order to document and accelerate this process,
a versioning system and a flexible format for specifying configurations
are desirable.

Two file formats for sharing LES setups have become adopted by multiple
groups over the past years. The KNMI parametrisation testbed format has
been used for LES modelling at observational supersites, whereas the
DEPHY format is a recent community effort where one of the goals was to
make existing case studies available to the LES community.

Lagtraj is a tool that aims to support and document the design of LES
case studies, and produce case studies using both of these formats. It
has been developed to use ERA-5 reanalysis data \citep{hersbach2020},
though with an eye to extending this to other input formats. As LES
modelling is particularly sensitive to the level of inversions in input
and nudging profiles, reanalyis data on model levels is used. Lagtraj
also aims to facilitate blending reanalysis and observational data.

This document describes the Lagtraj software in more detail. Section
\ref{sec:design} describes the code design, including input and output
formats. Section \ref{sec:examples} shows two case studies, one at a
continuous supersite and one using a Lagrangian reference frame, and
section \ref{sec:evaluation} describes both the way the code itself is
maintained and tested, as well as some examples of parameters that the
output of the code is sensitive to.

\section{Code design}\label{sec:design}

Lagtraj is organised in a modular fashion. The source code is organised
into directories that contain the code dealing with retrieving data,
setting up the Lagrangian trajectory or Eulerian time window, producing
the input (forcings) for LES models, and converting these inputs to
different formats. Lagtraj comes with input examples, unit tests, and
documentation. The code is python-based, and though it has several
dependencies, once these dependencies are installed the code does not
need further installation or compilation.

Its input files use the yaml format \citep{ben2009}, whereas data is
internally stored using xarray DataSets \citep{hoyer2017}. The latter
format was chosen as it provides for easy conversion from and to the
NetCDF format.

\subsection{Data download}

A download module has has been implemented for ERA5. This facilitates
creating requests for the data that is used in the creation of the LES
input, and also serves to submit and track requests.

The ERA5 data resides on the MARS archive and is retrieved through the
Copernicus Data Store \citep{raoult2017}. The data is retrieved in NetCDF format on a
regular latitude-longitude grid. For each day of the simulation, four
separate requests are created, which handle data on all model levels
and data on single levels (e.g. surface levels, vertically integrated
quantities) separately. The other distinction is between reanalysis
data that is available on an hourly basis, and data from the `forecast'
that corresponds to the reanalysis (some of the fields are only
available on an hourly basis in this forecast data).

The domain can be either specified or calculated using an existing
trajectory. Retrieving the files involves reinterpolation, which is
done on the Copernicus Data Store side, and the resolution for this
horizontal interpolation needs to be defined (we typically use 0.1
degrees). The download script places requests at the CDS data store.

Since it can take up to several days for model level data to become
available, there is the option to either automatically retry
downloading at a reqular interval or to retry downloading data when
requested by the user (the latter is useful on e.g. a laptop).
The download interface also detects when requested data should be
already available, and when it has been downloaded before but has been
data deleted.

\subsection{Trajectory calculations}

Trajectory calculations are performed using the winds at either a single
height or pressure level, or using a weighted average between two levels.
Trajectories are calculated based on the velocities at a single point.
An iterative strategy \citep{petterssen1956,sprenger2015} is used for time-stepping.
For forward trajectories, this looks as follows:

No substeps are currently used, although this could be changed.


\subsection{Forcing calculations}

The forcing calculations can be divided into 5 categories:
- Local profiles (of thermodynamic and dynamic variables, radiative
properties): for initialisation and nudging purposes. \\
- Mean profiles: as above, but averaged over a larger area. For a
domain over the ocean, this will likely result. Over land, it can be
desirable to use the local profiles in order to capture local effects
of land surface properties and topography. \\
- Advective tendencies (depend on gradients of the prognostic fields
and the relative speed of the Lagrangian trajectory compared to the wind
at each level). \\
- Surface properties (e.g. STTs or fluxes, albedo). \\
- Geostrophic winds (depend on pressure-gradients and latitude).. \\

In order to calculate these, a choice is required for the output
levels, averaging width for mean calculations and gradients, and any
masking that is applied. The masks can be used to filter out points
only over land or only over the ocean for the mean profiles
and calculations of gradients us treatment of topography. \\

The levels can be linearly or exponentially spaced in height. For the
latter option, the spacing is determined based on a desired number of
levels, domain top and the spacing near the surface.

Steffen interpolation (a monotonic method based on third-order
polynomials) is used to calculate the prognostic fields to height
levels, following \cite{yamaguchi2012}.

Gradients, which are used to determine geostrophic winds and advective
tendencies, can be based either on regression or on the values of the
edge of the domain. The use of a regression-based method usually
results in steeper gradients. For gradient calculations, a box of with
spefcified latitude and longitude differences is used. If regression is used,
a local coordinate system is set up around the central point of this box,
whereas for boundary values, latitude and longitude increments are
converted to distances.

\subsection{Input and output formats}

Input formats: \\
- ERA5. Download from CDS . \\
- ERA5 with dropsonde corrections \citep{bony2019} \\
- ICON/UM? \\

Output formats:
- ERA5 close to native, including auxiliary variables. \\

Driver formats:
- KNMI parametrisation testbed (KPT). \\
- DEPHY/iDEPHYx: the DEPHY format (impl\'ementation DEPHY avec extensions). \\

\section{Examples}\label{sec:examples}

- EUREC4A trajectories \cite{bony2017}. \\
- Single site (Cardington?) \\
- Include plots.

\section{Evaluation}\label{sec:evaluation}

\subsection{Continuous integration and unit testing.}

pytest \citep{okken2017}

\subsection{Sensitivity of trajectory and forcing to parameters}


\subsection{Validation against other codes}

Lagtraj has been compared against forcings that were derived for the
KNMI parameterisation testbed using the tools described in . This used
forecast (rather than reanalysis) data on pressure (rather than model)
levels. Check: is Lagtraj indeed less "noisy", as our initial
experiments suggested?

\section*{Things to look into}

- Test over land and ocean? \\
- Support for cfgrib?
- Anti-meridian and pole handling? \\
- Talk to Copernicus about how to best access their data? \\

\section{Discussion}

\section{Conclusions}

\bibliographystyle{apalike}
\bibliography{lagtraj_doc}

\end{document}

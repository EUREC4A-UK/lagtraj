\documentclass[a4paper,11pt]{article}

\usepackage{amsmath}
\usepackage{amssymb}
\usepackage{natbib}
\usepackage{graphicx}
\usepackage[a4paper,left=2cm,right=2cm,top=2cm,bottom=2cm]{geometry}

\title{Lagtraj: a tool for calculation LES forcings along trajectories}

\begin{document}

\maketitle

DRAFT VERSION

\section*{Abstract}

In recent years, Large-Eddy Simulations that span a length of
multiple days have become increasingly popular. These simulations are
often performed in the context of observational campaigns, or to study
the development of convection over multiple days at observational
supersites. Their main purpose is to simulate the behaviour of
convection and associatied process (microphysics, radiation) at the
scale of metres to a few hundred kilometres.

%~ For case studies over the
%~ ocean, the simulations often follow an air-mass in a reference frame
%~ that moves along with the boundary layer or lower cloud layer. For LES
%~ that uses doubly periodic boundary conditions, so-called large-scale
%~ forcings need to be prescribed. These include the effects of advection
%~ of air at levels where the flow velocity is different from the
%~ reference frame velocity, and optionally a nudging term that corrects
%~ for differences in the mean state between the LES and observations.

Lagtraj is a novel framework for calculation of the large-scale
forcings that these simulations require based on forecast or reanalysis
data, in some cases complemented by observational constraints. It aims
to streamline and the workflow that is required to perform routine
Large-Eddy Simulation and sensitivity studies based on realistic cases,
and ensure simulations are set up in a traceable fashion. Lagtraj has
been developed to serve as a community tool that can be further
extended with different input and output formats.

\section{Introduction}

- Developments in the use of Large-Eddy Simulation (LES): \\
** Development of LES for atmospheric science dates back to \cite{deardorff1970}. \\
** Due to the computational cost, LES was initially used for short
case studies. Initial conditions and forcings were often initialised in order
to make it easier to set up a simulation. \\
** Over the last two decades, Eulerian LES modelling for campaign
periods (e.g. TWP-ICE). Continuous super-site modelling: useful for
long-term systematic evaluation of parametrisation \cite{neggers2012}.
Routine LES operation to improve parametrisations
\cite{schalkwijk2015}, \cite{laar2019},\cite{gustafson2020}. \\
** Lagrangian case studies: for convection over the ocean, useful to
follow air in boundary layer or low clouds. Aim: capture development of
organisation. ASTEX \cite{bretherton1999}, \cite{roode2016},
\cite{tomassini2017}, \cite{mohrmann2019}, \cite{neggers2019}, NOAA
group. Experience using realistic forcings. \\
- Details of forcing matter, see e.g. \cite{smalley2019}. Replication
can be difficult. Sensitive to exact details of setup, such as
subsidence \citep{hohenegger2013,kurowski2020} and implementation of
surface fluxes cite \citep{stevens2001}. \\
- Replication is particularly important when comparing diffent LES
models. The time investment to produce the forcings can be an obstacle
to participation in an intercomparison. \\
- For replication, it is important for these configurations that the
workflow is traceable. The use of the NetCDF format (rather than a
text-based format) makes it easy to add meta-data. \\
- The forcings for LES intercomparison studies are often derived
through an iterative process. In order to document and accelerate this
process, a versioning system and a flexible format for specifying
configurations are desirable. \\
- Lagtraj is a tool that aims to support and document the design of LES
case studies. \\
- It was originally developed to use ERA-5 reanalysis data
\citep{hersbach2020}, though this is extendable. In particular, Lagtraj
aims to facilitate blending reanalysis and observational data. As LES
modelling is particularly sensitive to the level of inversions in input
and nudging profiles, reanalyis data on model levels is used. \\
- Two formats for sharing LES setups have become adopted by multiple
groups over the past years. In particular, the KNMI parametrisation
testbed format has been used for LES modelling at observational
supersites, whereas the DEPHY format is a recent community effort where
one of the goals was to make existing case studies available to the LES
community.
- Lagtraj is python-based, and though it has several dependencies, once
these dependencies are installed the code does not need further
installation or compilation. \\
- This document is outlined as follows. Section \ref{sec:design}
describes the code design, including input and output formats. Section
\ref{sec:examples} shows two case studies, one at a continuous supersite
and one using a Lagrangian reference frame, and section
\ref{sec:evaluation} describes both the way the code itself is
maintained and tested, as well as some examples of parameters that the
output of the code is sensitive to. \\

\section{Code design}\label{sec:design}

General code design: \\
Lagtraj is a modular code
Use of xarray \citep{hoyer2017}, yaml \citep{ben2009}

Steps \\
- Data download. This has been implemented for ERA5. The ERA5 data
resides on the MARS archive. Four types of data: model levels and
single levels, hourly reanalysis data and ERA5 hourly `forecast' data.
Requested as one file per day, currently using NetCDF format on regular
lat-lon grid. Domain needs to be specified or can be obtained from an
existing trajectory. This involves reinterpolation on ECMWF side,
horizontal interpolation needs to be defined. The download script
places requests at the CDS data store.
** Request can be left pending \\
** Situations dealts with: requested data already available, requested data deleted \\
- Trajectory calculations uses an iterative strategy \citep{petterssen1956,sprenger2015}. \\
** Options: use a single level, use a weighted average.
- Forcing calculations: \\
** Parameters: gradient method, levels, averaging width. \\
** Calculation of height levels.
** Use of Steffen interpolation to interpolate to height levels, following \cite{yamaguchi2012} \\
** Gradient calculation currently based on regular grid. Can use either a regression or edges. Use of masks and treatment of topography (what to do for land? ignore points). \\

\subsection{Formats}

Input formats: \\
- ERA5 (model level or pressure level?). Download from CDS \citep{raoult2017}. \\
- ERA5 with dropsonde corrections \citep{bony2019} \\
- ICON/UM? \\

Output formats:
- ERA5 close to native, including auxiliary variables. \\

Driver formats:
- KNMI parametrisation testbed (KPT) \\
- DEPHY/iDEPHYx: the DEPHY format (impl\'ementation DEPHY avec extensions). \\

\section{Examples}\label{sec:examples}

- EUREC4A trajectories \cite{bony2017} \\
- Single site (Cardington?, Julich?) \\
- Include plots

\section{Evaluation}\label{sec:evaluation}

\subsection{Continuous integration and unit testing.}

pytest \citep{okken2017}

\subsection{Sensitivity of trajectory and forcing to parameters}



\subsection{Validation against other codes}

Lagtraj has been compared against forcings that were derived for the KNMI parameterisation testbed using the tools described in .
This used forecast (rather than reanalysis) data on pressure (rather than model) levels.
Check: is Lagtraj indeed less "noisy", as our initial experiments suggested. can this be

\section*{Things to look into}

- Test over land and ocean? \\
- Support for cfgrib?
- Anti-meridian and pole handling? \\
- Talk to Copernicus about how to best access their data? \\

\section{Discussion}

\bibliographystyle{apalike}
\bibliography{lagtraj_doc}

\end{document}

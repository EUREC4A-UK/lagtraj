\documentclass[a4paper,11pt]{article}

\usepackage{amsmath}
\usepackage{amssymb}
\usepackage{natbib}
\usepackage{graphicx}
\usepackage[a4paper,left=2cm,right=2cm,top=2cm,bottom=2cm]{geometry}

\title{Lagtraj: a tool for calculation LES forcings along trajectories}

\begin{document}

\maketitle

DRAFT VERSION

\section*{Abstract}

In recent years, Large-Eddy Simulations that span a length of
multiple days have become increasingly popular. These simulations are
often performed in the context of observational campaigns, or to study
the development of convection over multiple days at observational
supersites. Their main purpose is to simulate the behaviour of
convection and associatied process (microphysics, radiation) at the
scale of metres to a few hundred kilometres.

%~ For case studies over the
%~ ocean, the simulations often follow an air-mass in a reference frame
%~ that moves along with the boundary layer or lower cloud layer. For LES
%~ that uses doubly periodic boundary conditions, so-called large-scale
%~ forcings need to be prescribed. These include the effects of advection
%~ of air at levels where the flow velocity is different from the
%~ reference frame velocity, and optionally a nudging term that corrects
%~ for differences in the mean state between the LES and observations.

Lagtraj is a novel framework for calculation of the large-scale
forcings that these simulations require based on forecast or reanalysis
data, in some cases complemented by observational constraints. It aims
to streamline the workflow that is required to perform routine
Large-Eddy Simulation and sensitivity studies based on realistic cases,
and ensure simulations are set up in a traceable fashion. Lagtraj has
been developed to serve as a community tool that can be further
extended with different input and output formats.

\section{Introduction}

Developments in the use of Large-Eddy Simulation (LES): \\
** Development of LES for atmospheric science dates back to \cite{deardorff1970}. \\
** Due to the computational cost, LES was initially used for short
case studies. Initial conditions and forcings were often initialised in order
to make it easier to set up a simulation. \\
** Over the last two decades, high-resolution convection-permitting
modelling and LES have been performed for intensive observational
campaigns, such as TWP-ICE \citep{fridlind2012}. Such studies are
particularly useful when detailed information about clouds or
turbulence are available. Another novel application of LES is within a
continuous framework around an observational super-site. This
super-site modelling has proven a useful strategy for long-term
systematic evaluation of parametrisations in NWP models
\cite{neggers2012}. Routine LES operation in such a context is useful
both for testing the LES itself (for example, its surface
parametrisation and its representation of clouds), and to help improve
and evaluate parameterisations in NWP and climat models
\cite{schalkwijk2015}, \cite{laar2019},\cite{gustafson2020}. For
modelling of case studies over land, often a Eulerian frame of
reference is preferred, both because the observations are often made
around a single location and because the height of the ground level can
vary along a flow trajectory.
** For convection over the ocean, it can be useful to follow air in
boundary layer or low clouds in a Lagrangian framework. This is
particularly relant for capturing the development of organisation and
clouds. Examples include \cite{bretherton1999}, \cite{roode2016},
\cite{tomassini2017}, \cite{mohrmann2019}, \cite{neggers2019} and work
from the NOAA group (add citation). The details of the exact setup matter, see e.g.
\cite{smalley2019}. Replication, is particularly important when comparing diffent LES
models, can be difficult, as it is sensitive to
exact details of setup, such as subsidence
\citep{hohenegger2013,kurowski2020} and implementation of surface
fluxes \citep{stevens2001}.
The time investment to produce the forcings can be an obstacle to
participation in an intercomparison. For replication, it is important
that the workflow to produce model setups is traceable. The use of a
fully automated framework and the retention of metadata regarding its
configuration are key here. The use of the
NetCDF format (rather than a text-based format) makes it easy to store
this metadata. The forcings for LES intercomparison studies are often derived
through an iterative process. In order to document and accelerate this
process, a versioning system and a flexible format for specifying
configurations are desirable.

Lagtraj is a tool that aims to support and document the design of LES
case studies, and produce case studies using both of these formats. It
has been developed to use ERA-5 reanalysis data \citep{hersbach2020},
though the input format could be changed and moreover Lagtraj aims to
facilitate blending reanalysis and observational data. As LES modelling
is particularly sensitive to the level of inversions in input and
nudging profiles, reanalyis data on model levels is used.

This document describes the Lagtraj software in more detail. Section \ref{sec:design}
describes the code design, including input and output formats. Section
\ref{sec:examples} shows two case studies, one at a continuous supersite
and one using a Lagrangian reference frame, and section
\ref{sec:evaluation} describes both the way the code itself is
maintained and tested, as well as some examples of parameters that the
output of the code is sensitive to.

\section{Code design}\label{sec:design}

Lagtraj is organised in a modular fashion. The source code is organised
into directories that containe the code dealing with retrieving data,
setting up the Lagrangian trajectory or Eulerian time window, producing
the input (forcings) for LES models, and converting these inputs to
different formats. Lagtraj comes with input examples, unit tests, and
documentation. The code is python-based, and though it has several
dependencies, once these dependencies are installed the code does not
need further installation or compilation.

Its input files use the yaml format \citep{ben2009}, whereas data is
internally stored using xarray DataSets \citep{hoyer2017}. The latter
format was chosen as it provides for easy conversion from and to the
NetCDF format.

\subsection{Data download.}

A download module has has been implemented for ERA5. This facilitates
creating requests for the data that is used in the creation of the LES
input, and also serves to submit and track requests.

The ERA5 data resides on the MARS archive. The data is retrieved in
NetCDF format on a regular latitude-longitude grid. For each day of the
simulation, four separate requests are created, which handle data on
all model levels and levels model on single levels (e.g. surface
levels, vertically integrated quantities) separately. The other
distinction is between reanalysis data that is available on an hourly
basis, and data from the `forecast' that corresponds to the reanalysis
(some of the fields are only available on an hourly basis in this
forecast data).

The domain can be either specified or calculated using an existing
trajectory. Retrieving the files involves reinterpolation, which is
done on the Copernicus Data Store side, and the resolution for this
horizontal interpolation needs to be defined (we typically use 0.1
degrees). The download script places requests at the CDS data store.

Request can be left pending \\ ** Situations dealts with: requested
data already available, requested data deleted \\

Trajectory calculations uses an iterative strategy \citep{petterssen1956,sprenger2015}. \\
** Options: use a single level, use a weighted average.

Forcing calculations: \\
** Parameters: gradient method, levels, averaging width. \\
** Calculation of height levels.
** Use of Steffen interpolation to interpolate to height levels, following \cite{yamaguchi2012} \\
** Gradient calculation currently based on regular grid. Can use either a regression or edges. Use of masks and treatment of topography (what to do for land? ignore points). \\

\subsection{Formats}

Two file formats for sharing LES setups have become adopted by multiple
groups over the past years. The KNMI parametrisation
testbed format has been used for LES modelling at observational
supersites, whereas the DEPHY format is a recent community effort where
one of the goals was to make existing case studies available to the LES
community.

Input formats: \\
- ERA5 (model level or pressure level?). Download from CDS \citep{raoult2017}. \\
- ERA5 with dropsonde corrections \citep{bony2019} \\
- ICON/UM? \\

Output formats:
- ERA5 close to native, including auxiliary variables. \\

Driver formats:
- KNMI parametrisation testbed (KPT) \\
- DEPHY/iDEPHYx: the DEPHY format (impl\'ementation DEPHY avec extensions). \\

\section{Examples}\label{sec:examples}

- EUREC4A trajectories \cite{bony2017} \\
- Single site (Cardington?, Julich?) \\
- Include plots

\section{Evaluation}\label{sec:evaluation}

\subsection{Continuous integration and unit testing.}

pytest \citep{okken2017}

\subsection{Sensitivity of trajectory and forcing to parameters}



\subsection{Validation against other codes}

Lagtraj has been compared against forcings that were derived for the KNMI parameterisation testbed using the tools described in .
This used forecast (rather than reanalysis) data on pressure (rather than model) levels.
Check: is Lagtraj indeed less "noisy", as our initial experiments suggested. can this be

\section*{Things to look into}

- Test over land and ocean? \\
- Support for cfgrib?
- Anti-meridian and pole handling? \\
- Talk to Copernicus about how to best access their data? \\

\section{Discussion}

\bibliographystyle{apalike}
\bibliography{lagtraj_doc}

\end{document}

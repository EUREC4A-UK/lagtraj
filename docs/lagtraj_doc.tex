\documentclass[a4paper,11pt]{article}

\usepackage{amsmath}
\usepackage{amssymb}
\usepackage{natbib}
\usepackage{graphicx}

\title{Lagtraj: a tool for calculation LES forcings along trajectories}

\begin{document}

\maketitle

DRAFT VERSION

\section*{Abstract}

In recent years, Large-Eddy Simulation (LES) that spans a length of
multiple days have become increasingly popular. These simulations are
often performed in the context of observational campaigns, or to study
the development of convection over multiple days at observational
supersites. Their main purpose is to simulate the behaviour of
convection and associatied process (microphysics, radiation) at the
scale of metres to a few hundred kilometres.

%~ For case studies over the
%~ ocean, the simulations often follow an air-mass in a reference frame
%~ that moves along with the boundary layer or lower cloud layer. For LES
%~ that uses doubly periodic boundary conditions, so-called large-scale
%~ forcings need to be prescribed. These include the effects of advection
%~ of air at levels where the flow velocity is different from the
%~ reference frame velocity, and optionally a nudging term that corrects
%~ for differences in the mean state between the LES and observations.

Lagtraj is a novel framework for calculation of the large-scale
forcings that these simulations require based on forecast or reanalysis
data, in some cases complemented by observational constraints. It aims
to streamline and the workflow that is required to perform routine LES
and sensitivity studies based on realistic cases, and ensure
simulations are set up in a traceable fashion. Lagtraj has been
developed to serve as a community tool that can be further extended
with different input and output formats.

\section{Introduction}

- Developments in the use of Large-Eddy Simulation: \\
** Continuous super-site modelling: useful for long-term systematic
evaluation of parametrisation \cite{neggers2012}. Routine LES operation
to improve parametrisations
\cite{schalkwijk2015},\cite{laar2019},\cite{gustafson2020}. \\
** Lagrangian case studies: for convection over the ocean, useful to
follow air in boundary layer or low clouds. Aim: capture development of
organisation. ASTEX \cite{bretherton1999}, \cite{roode2016},
\cite{tomassini2017}, \cite{mohrmann2019}, \cite{neggers2019}, NOAA
group. Experience using realistic forcings. \\
** Details of forcing matter, see e.g. \cite{smalley2019}. \\
- Replication can be difficult. Sensitive to exact details of setup, such as subsidence \citep{kurowski2020} and implementation of surface fluxes cite \citep{stevens2001}. \\
- Difficulty comparing models for these situations, and time investment to make models work. \\
- Importance of traceable configurations: even for idealised cases, there have often been small but important variations in the forcings used (mention BOMEX?). \\
- Traditionally, text files have been used to share configurations. \\
- Need for flexible configurable tool. \\
- Use of best available data \cite{hersbach2020} (model levels)? Need to represent inversions. \\
- Extendibility (e.g. merging in observations) \\
- Lagtraj: python based, dependencies (code is plug-and-go) \\

\section{Code design}

Steps \\
- Data download \\
- Trajectory calculations uses an iterative strategy \citep{petterssen1956,sprenger2015} \\
- Forcing calculations: \\
** Parameters \\
** Use of Steffen interpolation \citep{yamaguchi2012} \\
** Gradient calculation
** Cite references in Wernli about iterative procedure for trajectories \\
- Format conversion and additional parameters \\

Other factors: \\
- Topography \\
- Land-sea mask \\

Code design: \\
Modular, using xarray \citep{hoyer2017}, yaml \citep{ben2009}

\subsection{Formats}

Input formats: \\
- ERA5 (model level or pressure level?). Download from CDS \citep{raoult2017} \\
- ERA5 with dropsonde corrections \citep{bony2019} \\
- ICON/UM? \\

Output formats:
- ERA5 close to native, including auxiliary variables. \\
- KNMI parametrisation testbed (KPT) \\
- DEPHY/iDEPHYx: the DEPHY format (implémentation DEPHY avec extensions). \\

\section{Examples}

- EUREC4A trajectories \cite{bony2017} \\
- Single site (Cardington?, Julich?) \\
- Include plots

\section{Evaluation}

\subsection{Continuous integration and unit testing.}

pytest \citep{okken2017}

\subsection{Sensitivity of trajectory and forcing to parameters}



\subsection{Validation against other codes}

Lagtraj has been compared against forcings that were derived for the KNMI parameterisation testbed using the tools described in .
This used forecast (rather than reanalysis) data on pressure (rather than model) levels.
Check: is Lagtraj indeed less "noisy", as our initial experiments suggested. can this be

\section*{Things to look into}

- Test over land and ocean? \\
- Anti-meridian and pole handling? \\
- Talk to Copernicus about how to best access their data? \\

\section{Discussion}

\bibliographystyle{apalike}
\bibliography{lagtraj_doc}

\end{document}
